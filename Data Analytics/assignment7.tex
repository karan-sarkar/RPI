\documentclass[]{article}
\usepackage{lmodern}
\usepackage{amssymb,amsmath}
\usepackage{ifxetex,ifluatex}
\usepackage{fixltx2e} % provides \textsubscript
\ifnum 0\ifxetex 1\fi\ifluatex 1\fi=0 % if pdftex
  \usepackage[T1]{fontenc}
  \usepackage[utf8]{inputenc}
\else % if luatex or xelatex
  \ifxetex
    \usepackage{mathspec}
  \else
    \usepackage{fontspec}
  \fi
  \defaultfontfeatures{Ligatures=TeX,Scale=MatchLowercase}
\fi
% use upquote if available, for straight quotes in verbatim environments
\IfFileExists{upquote.sty}{\usepackage{upquote}}{}
% use microtype if available
\IfFileExists{microtype.sty}{%
\usepackage{microtype}
\UseMicrotypeSet[protrusion]{basicmath} % disable protrusion for tt fonts
}{}
\usepackage[margin=1in]{geometry}
\usepackage{hyperref}
\hypersetup{unicode=true,
            pdftitle={Assignment 7},
            pdfborder={0 0 0},
            breaklinks=true}
\urlstyle{same}  % don't use monospace font for urls
\usepackage{color}
\usepackage{fancyvrb}
\newcommand{\VerbBar}{|}
\newcommand{\VERB}{\Verb[commandchars=\\\{\}]}
\DefineVerbatimEnvironment{Highlighting}{Verbatim}{commandchars=\\\{\}}
% Add ',fontsize=\small' for more characters per line
\usepackage{framed}
\definecolor{shadecolor}{RGB}{248,248,248}
\newenvironment{Shaded}{\begin{snugshade}}{\end{snugshade}}
\newcommand{\KeywordTok}[1]{\textcolor[rgb]{0.13,0.29,0.53}{\textbf{#1}}}
\newcommand{\DataTypeTok}[1]{\textcolor[rgb]{0.13,0.29,0.53}{#1}}
\newcommand{\DecValTok}[1]{\textcolor[rgb]{0.00,0.00,0.81}{#1}}
\newcommand{\BaseNTok}[1]{\textcolor[rgb]{0.00,0.00,0.81}{#1}}
\newcommand{\FloatTok}[1]{\textcolor[rgb]{0.00,0.00,0.81}{#1}}
\newcommand{\ConstantTok}[1]{\textcolor[rgb]{0.00,0.00,0.00}{#1}}
\newcommand{\CharTok}[1]{\textcolor[rgb]{0.31,0.60,0.02}{#1}}
\newcommand{\SpecialCharTok}[1]{\textcolor[rgb]{0.00,0.00,0.00}{#1}}
\newcommand{\StringTok}[1]{\textcolor[rgb]{0.31,0.60,0.02}{#1}}
\newcommand{\VerbatimStringTok}[1]{\textcolor[rgb]{0.31,0.60,0.02}{#1}}
\newcommand{\SpecialStringTok}[1]{\textcolor[rgb]{0.31,0.60,0.02}{#1}}
\newcommand{\ImportTok}[1]{#1}
\newcommand{\CommentTok}[1]{\textcolor[rgb]{0.56,0.35,0.01}{\textit{#1}}}
\newcommand{\DocumentationTok}[1]{\textcolor[rgb]{0.56,0.35,0.01}{\textbf{\textit{#1}}}}
\newcommand{\AnnotationTok}[1]{\textcolor[rgb]{0.56,0.35,0.01}{\textbf{\textit{#1}}}}
\newcommand{\CommentVarTok}[1]{\textcolor[rgb]{0.56,0.35,0.01}{\textbf{\textit{#1}}}}
\newcommand{\OtherTok}[1]{\textcolor[rgb]{0.56,0.35,0.01}{#1}}
\newcommand{\FunctionTok}[1]{\textcolor[rgb]{0.00,0.00,0.00}{#1}}
\newcommand{\VariableTok}[1]{\textcolor[rgb]{0.00,0.00,0.00}{#1}}
\newcommand{\ControlFlowTok}[1]{\textcolor[rgb]{0.13,0.29,0.53}{\textbf{#1}}}
\newcommand{\OperatorTok}[1]{\textcolor[rgb]{0.81,0.36,0.00}{\textbf{#1}}}
\newcommand{\BuiltInTok}[1]{#1}
\newcommand{\ExtensionTok}[1]{#1}
\newcommand{\PreprocessorTok}[1]{\textcolor[rgb]{0.56,0.35,0.01}{\textit{#1}}}
\newcommand{\AttributeTok}[1]{\textcolor[rgb]{0.77,0.63,0.00}{#1}}
\newcommand{\RegionMarkerTok}[1]{#1}
\newcommand{\InformationTok}[1]{\textcolor[rgb]{0.56,0.35,0.01}{\textbf{\textit{#1}}}}
\newcommand{\WarningTok}[1]{\textcolor[rgb]{0.56,0.35,0.01}{\textbf{\textit{#1}}}}
\newcommand{\AlertTok}[1]{\textcolor[rgb]{0.94,0.16,0.16}{#1}}
\newcommand{\ErrorTok}[1]{\textcolor[rgb]{0.64,0.00,0.00}{\textbf{#1}}}
\newcommand{\NormalTok}[1]{#1}
\usepackage{graphicx,grffile}
\makeatletter
\def\maxwidth{\ifdim\Gin@nat@width>\linewidth\linewidth\else\Gin@nat@width\fi}
\def\maxheight{\ifdim\Gin@nat@height>\textheight\textheight\else\Gin@nat@height\fi}
\makeatother
% Scale images if necessary, so that they will not overflow the page
% margins by default, and it is still possible to overwrite the defaults
% using explicit options in \includegraphics[width, height, ...]{}
\setkeys{Gin}{width=\maxwidth,height=\maxheight,keepaspectratio}
\IfFileExists{parskip.sty}{%
\usepackage{parskip}
}{% else
\setlength{\parindent}{0pt}
\setlength{\parskip}{6pt plus 2pt minus 1pt}
}
\setlength{\emergencystretch}{3em}  % prevent overfull lines
\providecommand{\tightlist}{%
  \setlength{\itemsep}{0pt}\setlength{\parskip}{0pt}}
\setcounter{secnumdepth}{0}
% Redefines (sub)paragraphs to behave more like sections
\ifx\paragraph\undefined\else
\let\oldparagraph\paragraph
\renewcommand{\paragraph}[1]{\oldparagraph{#1}\mbox{}}
\fi
\ifx\subparagraph\undefined\else
\let\oldsubparagraph\subparagraph
\renewcommand{\subparagraph}[1]{\oldsubparagraph{#1}\mbox{}}
\fi

%%% Use protect on footnotes to avoid problems with footnotes in titles
\let\rmarkdownfootnote\footnote%
\def\footnote{\protect\rmarkdownfootnote}

%%% Change title format to be more compact
\usepackage{titling}

% Create subtitle command for use in maketitle
\newcommand{\subtitle}[1]{
  \posttitle{
    \begin{center}\large#1\end{center}
    }
}

\setlength{\droptitle}{-2em}

  \title{Assignment 7}
    \pretitle{\vspace{\droptitle}\centering\huge}
  \posttitle{\par}
    \author{}
    \preauthor{}\postauthor{}
    \date{}
    \predate{}\postdate{}
  

\begin{document}
\maketitle

\section{Section 1}\label{section-1}

We will be using the wine datasets from the UCI Machine Learning dataset
collection.

\begin{Shaded}
\begin{Highlighting}[]
\NormalTok{red <-}\StringTok{ }\KeywordTok{read.csv}\NormalTok{(}\StringTok{"winequality-red.csv"}\NormalTok{)}
\NormalTok{white <-}\StringTok{ }\KeywordTok{read.csv}\NormalTok{(}\StringTok{"winequality-white.csv"}\NormalTok{)}
\KeywordTok{library}\NormalTok{(RColorBrewer)}
\NormalTok{colors <-}\StringTok{ }\KeywordTok{brewer.pal}\NormalTok{(}\DecValTok{6}\NormalTok{, }\StringTok{"Spectral"}\NormalTok{)}
\NormalTok{red}\OperatorTok{$}\NormalTok{color <-}\StringTok{ }\NormalTok{colors[red}\OperatorTok{$}\NormalTok{quality}\OperatorTok{-}\DecValTok{2}\NormalTok{]}
\NormalTok{white}\OperatorTok{$}\NormalTok{color <-}\StringTok{ }\NormalTok{colors[white}\OperatorTok{$}\NormalTok{quality}\OperatorTok{-}\DecValTok{2}\NormalTok{]}
\end{Highlighting}
\end{Shaded}

\begin{Shaded}
\begin{Highlighting}[]
\KeywordTok{pairs}\NormalTok{(red[,}\DecValTok{1}\OperatorTok{:}\DecValTok{11}\NormalTok{], }\DataTypeTok{col =}\NormalTok{ red}\OperatorTok{$}\NormalTok{color)}
\end{Highlighting}
\end{Shaded}

\includegraphics{assignment7_files/figure-latex/unnamed-chunk-2-1.pdf}

\begin{Shaded}
\begin{Highlighting}[]
\KeywordTok{pairs}\NormalTok{(white[,}\DecValTok{1}\OperatorTok{:}\DecValTok{11}\NormalTok{], }\DataTypeTok{col =}\NormalTok{ white}\OperatorTok{$}\NormalTok{color)}
\end{Highlighting}
\end{Shaded}

\includegraphics{assignment7_files/figure-latex/unnamed-chunk-2-2.pdf}

Examining the scatterplots shows that several variables are corelated.
Factor analysis or principal components analysis might provide some
insights into latent variables behind the correlations. We can also see
that none of the variables are too severely skewed. THere are not
outliers in the data, so nothing has to be removed.

\begin{Shaded}
\begin{Highlighting}[]
\NormalTok{red.pca <-}\StringTok{ }\KeywordTok{prcomp}\NormalTok{(red[,}\DecValTok{1}\OperatorTok{:}\DecValTok{11}\NormalTok{])}
\NormalTok{white.pca <-}\StringTok{ }\KeywordTok{prcomp}\NormalTok{(white[,}\DecValTok{1}\OperatorTok{:}\DecValTok{11}\NormalTok{])}
\KeywordTok{plot}\NormalTok{(red.pca}\OperatorTok{$}\NormalTok{x[,}\DecValTok{1}\NormalTok{], red.pca}\OperatorTok{$}\NormalTok{x[,}\DecValTok{2}\NormalTok{],}\DataTypeTok{col=}\NormalTok{red}\OperatorTok{$}\NormalTok{color)}
\end{Highlighting}
\end{Shaded}

\includegraphics{assignment7_files/figure-latex/unnamed-chunk-3-1.pdf}

\begin{Shaded}
\begin{Highlighting}[]
\KeywordTok{plot}\NormalTok{(white.pca}\OperatorTok{$}\NormalTok{x[,}\DecValTok{1}\NormalTok{], white.pca}\OperatorTok{$}\NormalTok{x[,}\DecValTok{2}\NormalTok{],}\DataTypeTok{col=}\NormalTok{white}\OperatorTok{$}\NormalTok{color)}
\end{Highlighting}
\end{Shaded}

\includegraphics{assignment7_files/figure-latex/unnamed-chunk-3-2.pdf}
Looking a the Principal components plots, we can see that certain
quality levels are clustered together. However the boundary between
quality levels is fuzzy. Therefore, some sort of nonlinear regression
seems like a good idea. Because the distinctions between levels is
arbitary, linear regression does not really make sense. Moreover,
because the boundaries are not distinct, clustering is not likely to
work.

\section{Section 2}\label{section-2}

We will be attempting to predict wine quality for white wine. We will do
validation using a training set and a training set. We will use training
set and test set validation because it is relatively simple to
implement.

\begin{Shaded}
\begin{Highlighting}[]
\NormalTok{train <-}\StringTok{ }\KeywordTok{sample}\NormalTok{(}\KeywordTok{nrow}\NormalTok{(white), }\DecValTok{2000}\NormalTok{)}
\NormalTok{white.train <-}\StringTok{ }\NormalTok{white[train,]}
\NormalTok{white.test <-}\StringTok{ }\NormalTok{white[}\OperatorTok{-}\NormalTok{train,]}
\end{Highlighting}
\end{Shaded}

\begin{Shaded}
\begin{Highlighting}[]
\NormalTok{lm <-}\StringTok{ }\KeywordTok{lm}\NormalTok{(}\KeywordTok{as.numeric}\NormalTok{(quality)}\OperatorTok{~}\NormalTok{., white.train[}\DecValTok{1}\OperatorTok{:}\DecValTok{12}\NormalTok{])}
\KeywordTok{summary}\NormalTok{(lm)}
\end{Highlighting}
\end{Shaded}

\begin{verbatim}
## 
## Call:
## lm(formula = as.numeric(quality) ~ ., data = white.train[1:12])
## 
## Residuals:
##     Min      1Q  Median      3Q     Max 
## -3.2700 -0.4836 -0.0648  0.4316  2.5185 
## 
## Coefficients:
##                        Estimate Std. Error t value Pr(>|t|)    
## (Intercept)           9.927e+01  2.450e+01   4.051 5.29e-05 ***
## fixed.acidity        -6.257e-03  3.048e-02  -0.205   0.8373    
## volatile.acidity     -1.740e+00  1.704e-01 -10.211  < 2e-16 ***
## citric.acid           2.020e-01  1.485e-01   1.361   0.1738    
## residual.sugar        6.209e-02  1.049e-02   5.920 3.79e-09 ***
## chlorides             2.902e-01  7.486e-01   0.388   0.6983    
## free.sulfur.dioxide   2.789e-03  1.292e-03   2.158   0.0311 *  
## total.sulfur.dioxide -4.269e-05  5.787e-04  -0.074   0.9412    
## density              -9.837e+01  2.490e+01  -3.950 8.08e-05 ***
## pH                    3.882e-01  1.563e-01   2.483   0.0131 *  
## sulphates             7.235e-01  1.561e-01   4.635 3.80e-06 ***
## alcohol               2.609e-01  3.206e-02   8.140 6.88e-16 ***
## ---
## Signif. codes:  0 '***' 0.001 '**' 0.01 '*' 0.05 '.' 0.1 ' ' 1
## 
## Residual standard error: 0.7362 on 1988 degrees of freedom
## Multiple R-squared:  0.2803, Adjusted R-squared:  0.2763 
## F-statistic: 70.39 on 11 and 1988 DF,  p-value: < 2.2e-16
\end{verbatim}

\begin{Shaded}
\begin{Highlighting}[]
\NormalTok{lm.preds <-}\StringTok{ }\KeywordTok{predict}\NormalTok{(lm, white.test)}
\KeywordTok{plot}\NormalTok{(lm.preds, white.test}\OperatorTok{$}\NormalTok{quality)}
\end{Highlighting}
\end{Shaded}

\includegraphics{assignment7_files/figure-latex/unnamed-chunk-5-1.pdf}

\begin{Shaded}
\begin{Highlighting}[]
\NormalTok{lm.mse <-}\StringTok{ }\KeywordTok{sum}\NormalTok{((lm.preds }\OperatorTok{-}\StringTok{ }\NormalTok{white.test}\OperatorTok{$}\NormalTok{quality)}\OperatorTok{^}\DecValTok{2}\NormalTok{)}
\NormalTok{lm.mse}
\end{Highlighting}
\end{Shaded}

\begin{verbatim}
## [1] 1696.333
\end{verbatim}

We construct a simple linear model to predict wine quality. We are using
a linear model as a baseline so we can judge the accuracy of the random
forest model that we will use later. Moreover, the linear model is very
easy to interpret. We can see that factors like volatile acidity and
alcohol are statistically significant. The small p-value associated with
the F-statistic indicates that the linear model is statistically
significant. The negative coefficient on density shows that low
densities are prefered while the positive coefficient on alcohol shows
that high alcohol levels are preferred. The scatterplot of predicted
quality vs actual quality shows a positive trend. However, it reveals
that our model is not very precise. We get a total mean squared error of
1638 on the test data.

\begin{Shaded}
\begin{Highlighting}[]
\KeywordTok{library}\NormalTok{(randomForest)}
\end{Highlighting}
\end{Shaded}

\begin{verbatim}
## randomForest 4.6-14
\end{verbatim}

\begin{verbatim}
## Type rfNews() to see new features/changes/bug fixes.
\end{verbatim}

\begin{Shaded}
\begin{Highlighting}[]
\KeywordTok{library}\NormalTok{(forestFloor)}
\NormalTok{rf =}\StringTok{ }\KeywordTok{randomForest}\NormalTok{(}
\NormalTok{  quality}\OperatorTok{~}\NormalTok{.,}
\NormalTok{  white.train[}\DecValTok{1}\OperatorTok{:}\DecValTok{12}\NormalTok{],}
  \DataTypeTok{keep.inbag =} \OtherTok{TRUE}\NormalTok{, }
  \DataTypeTok{importance =} \OtherTok{TRUE}\NormalTok{,  }
  \DataTypeTok{mtry =} \DecValTok{3}\NormalTok{,}
  \DataTypeTok{prox =} \OtherTok{TRUE}\NormalTok{,}
  \DataTypeTok{ntree =} \DecValTok{1000}\NormalTok{,}
\NormalTok{)}
\NormalTok{rf.preds <-}\StringTok{ }\KeywordTok{predict}\NormalTok{(rf, white.test)}
\KeywordTok{plot}\NormalTok{(rf.preds, white.test}\OperatorTok{$}\NormalTok{quality)}
\end{Highlighting}
\end{Shaded}

\includegraphics{assignment7_files/figure-latex/unnamed-chunk-6-1.pdf}

\begin{Shaded}
\begin{Highlighting}[]
\NormalTok{rf.mse <-}\StringTok{ }\KeywordTok{sum}\NormalTok{((rf.preds }\OperatorTok{-}\StringTok{ }\NormalTok{white.test}\OperatorTok{$}\NormalTok{quality)}\OperatorTok{^}\DecValTok{2}\NormalTok{)}
\NormalTok{rf.mse}
\end{Highlighting}
\end{Shaded}

\begin{verbatim}
## [1] 1304.367
\end{verbatim}

\begin{Shaded}
\begin{Highlighting}[]
\NormalTok{ff =}\StringTok{ }\KeywordTok{forestFloor}\NormalTok{(}
  \DataTypeTok{rf =}\NormalTok{ rf ,       }\CommentTok{# mandatory}
  \DataTypeTok{X =}\NormalTok{ white.train,              }\CommentTok{# mandatory}
  \DataTypeTok{calc_np =} \OtherTok{FALSE}\NormalTok{,    }\CommentTok{# TRUE or FALSE both works, makes no difference}
  \DataTypeTok{binary_reg =} \OtherTok{FALSE}  \CommentTok{# takes no effect here when rfo$type="regression"}
\NormalTok{)}
\KeywordTok{plot}\NormalTok{(ff,  }\DataTypeTok{col =}\NormalTok{ white.train}\OperatorTok{$}\NormalTok{color  ,                 }\CommentTok{# forestFloor object}
     \DataTypeTok{orderByImportance=}\OtherTok{FALSE}    \CommentTok{# if TRUE index sequence by importance, else by X column  }
\NormalTok{)}
\end{Highlighting}
\end{Shaded}

\begin{verbatim}
## [1] "compute goodness-of-fit with leave-one-out k-nearest neighbor(guassian weighting), kknn package"
\end{verbatim}

\includegraphics{assignment7_files/figure-latex/unnamed-chunk-6-2.pdf}

We use a random forest model to predict wine quality. We see that the
test mean squared error of 1228 is lower than the linear model's test
mean squared error of 1638. This improvement in test accuracy over a
linear model we previously determined to be statistically significant
shows that the random forest model is also statistically significant. We
are using a random forest model because it is very good at making
predictions based on quantitative predictors. Unlike the linear model,
the random forest model can handle nonlinear trends. For example, in the
forest floor plots, we see that the optimal value of total sulfur
dioxide is around 100 not at the extremes of the range.On the other
hand, we can see that the sulphates variable is not very important. The
scatterplot shows that the random forest is still an inexact measure of
wine quality.

\section{Section 3}\label{section-3}

\begin{Shaded}
\begin{Highlighting}[]
\KeywordTok{varImpPlot}\NormalTok{(rf)}
\end{Highlighting}
\end{Shaded}

\includegraphics{assignment7_files/figure-latex/unnamed-chunk-7-1.pdf}

\begin{Shaded}
\begin{Highlighting}[]
\NormalTok{randomForest}\OperatorTok{::}\KeywordTok{MDSplot}\NormalTok{(rf, }\DataTypeTok{fac =} \KeywordTok{as.factor}\NormalTok{(white.train}\OperatorTok{$}\NormalTok{quality), }\DataTypeTok{k=}\DecValTok{3}\NormalTok{)}
\end{Highlighting}
\end{Shaded}

\includegraphics{assignment7_files/figure-latex/unnamed-chunk-7-2.pdf}

The random forest models does not provide us a very tight fit that is
super useful for predicting quality. The proximity plots show us that
the qualites do not separate perfectly. However, we can see what
variables lead to high wine quality. In particular, we can see that
lower densitiy and higher alcohol in general lead to higher scores.
Moreover, we can see an optimal range for citric acid. Understanding the
optimal values for each factor allows us to make decisions in that
winemakers could improve wine my finetuning these variables. A winemaker
could try to set target citric acid levels among other things because
these critical regions are statistically significant. On ther other
hand, using this model to predict quality is not reccomended because the
variance within qualities is so great.


\end{document}
